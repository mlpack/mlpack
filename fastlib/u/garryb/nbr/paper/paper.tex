
Big big big big big

my paper - parallel algorithms with application...


My claims:
  Generalized N-body Problems have locality
  Not trivial to parallelize, but their generalization makes parallelization
     trivial

What I must show:
 - characterize what dual-tree algorithms are
 - what are query/reference problems, and what is a pure pairwise map-reduce
 - first: characterize problems with global results, and go over intrinsic
 and extrinsic prunes


INTRO

  dual tree algorithms are x


----------------------------------------------------------------------------

P02 What is a generalized N-body problem?

----------------------------------------------------------------------------

P00 What is a dual tree algorithm?  Why is it useful?
dep: P02

----------------------------------------------------------------------------

P01 What are we showing?
dep: P00, P02

We find it is possible to express many dual-tree algorithms in a standard programming model.
Such a model allows flexible implementations, such that many different expansion patterns, including parallel ones, are allowed.
Our particular choices are motivated by ongoing yet unpublished work to formalize the characteristics of all generalized $N$-body problems.

----------------------------------------------------------------------------

P03 What is a global reduction and how do we generalize?
dep: P01

\subsection{Global reductions}

A global reduction computes a single scalar or relatively small vector of results.
These results are the application of a single commutative and associative operator to all pairs of inputs.
The simplest form of global reduction can be expressed as a pure recurrence relation, where all subcomputations are entirely independent.
A notable example is the two-point correlation problem:

$$\TWOPT(X, Y) = \sum{(a, b) \in X \times Y)} I(d(a, b) < r)$$

where $I$ is the indicator function, $d(a, b)$ is a distance metric, and $r$ is a radius, and both $X$ and $Y$ are initially the same data set.
Suppose then we can partition any set $X$ into two subsets $A_L$ and $A_R$ both of cardinality $\Theta(|X|)$, such as in a $kd$-tree.
Since $\TWOPT$ is a reduction, we can represent a simple recurrence relation:

% \left{ \begin{array}{lr} \\ \end{array} \right.

\begin{equationaray}
  \TWOPT(X, Y)
             &=& 0 \if d^\min(X, Y) \geq r
  \\
             &=& |X||Y| \if d^\max(X, Y) \leq r
  \\
             &=& \TWOPT(X_L, Y) + \TWOPT(X_R, Y)
  \\
             &=& \TWOPT(X, Y_L) + \TWOPT(X, Y_R)
\end{equationarray}

where $d^\max(X, Y)$ is the maximum distance between any two points in $X$ and $Y$, likewise for $d^\min$.
Computation of actual the maximum distance is not a trivial problem.
However, summary statistics allow the constant-time computation of an upper bound for the maximum distance.
In a $kd$-tree, one suitable summary statistic is the bounding hyper-rectangle for each node:

$d^\max(X, Y) \leq d^\max_{\hrect}(\sigma_{\hrect}(X), \sigma_{\hrect}(Y))$

Here we introduce the $\sigma$ meta-function, which is a summary statistic.
The $\sigma_{\hrect}$ is a statistic function which is used by virtually all dual-tree problems using $kd$-tree.
Two-point correlation additionally uses the cardinality statistic ($|X|$).
Series-expansion algorithms may require the mean, variance, or other moment functions of the data.

From this logic, we can generalize the remainder of the recurrence relation for simple global reduction problems.
We propose that problems of the form:

\begin{equation}
  \gamma(X, Y) = \oplus_{(a, b) \in X x Y} f(a,b)
\end{equation}

where $\oplus$ is commutative and associative, can often be expressed as:

\begin{equationaray}
  \gamma(X, Y)
             &=& \delta(\sigma(X), \sigma(Y)) \if C(\sigma(X), \sigma(Y))
  \\
             &=& \gamma(X_L, Y) \oplus_\gamma \gamma(X_R, Y)
  \\
             &=& \gamma(X, B_L) \oplus_\gamma \gamma(X, B_R)
\label{eqn:global_intrinsic}
\end{equationarray}

We then define a formal model for simple global reduction problems, in terms of the following operations:

\begin{itemize}
\item $\gamma(X, Y)$ - The result of a global reduction.
\item $\sigma(X)$ - Summary statistics over a range of data.
      The specialized operator $\oplus_{sigma}$ may be used to merge values of two subsets subsets: $\sigma(X) = \sigma(X_L) + \sigma(X_R)$.
      In the case of bounding hyper-rectangles, $\oplus_{sigma_\hrect}$ corresponds to an element-wise minimum and maximum.
\item $\delta(\sigma(X), \sigma(Y))$ - The estimated contribution of a sub-computation using only summary statistics.
\item $C(\sigma(X), \sigma(Y))$ - An indicator if a prune is possible, and that the subproblem $\gamma(X, Y)$ can be substituted by $\delta(\sigma(X), \sigma(Y))$.
\item $\oplus$ - The reduction operator.
\end{itemize}

The previous functions, combined with the generalized recurrence relation, completely define the problem at hand
\footnote{In practice, very small subsets may be computed exhaustively to reduce the overhead of treating individual points as entire subsets.}

One can note that the recurrence relationship is not specific about exactly how the expression is expanded.
The recurrence permits many types of expansion, such as depth-first, breadth-first, priority-queue, and importantly parallel expansions.
Thus, by separating the computation of the recurrence relationship from the problem-specific details, parallelization of all these algorithms requires only the up-front cost of a single parallel execution mechanism.

The model shown so far is very limited.
First, we later discuss additions to allow for efficient computation of query-reference problems such as all-nearest-neighbors and density estimation.
Additionally, many algorithms, such as nearest-neighbor, classification problems, and approximate density estimates, require information about previous pairwise computation in order to determine whether pruning is possible.
Nonetheless, this simple model leads to effective parallelization of problems such as two-point correlation\footnote{WALDO list more}.

----------------------------------------------------------------------------

P04 What is a query-reference problem?
dep: P01

A query-reference problem considers computes a result independently for each point in a query set.
These problems take on the general form $\map_{q \in Q} \oplus_{r \in R}$.

Although each query is independent, speedup is achieved by pruning for {\it an entire set} of reference points for a subset of the query set.
Distinct .


----------------------------------------------------------------------------

P05 Are we better than mapreduce?

----------------------------------------------------------------------------




Ongoing unpublished research attempts to
characterize all generalized N-body problems with rigorous mathematical
formality; however, such rigorous formalization is outside the scope of this
paper.  Here, we postulate that all dual-tree algorithms fall into one or
both of two categories: a single scalar or vector result summarizing all
pairs of inputs, or an independent result for each query point computed
against the entire reference set.

The former category can be generalized by a few custom functions that are
unique to the problem at hand, with many possible implementations.







