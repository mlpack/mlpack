\documentclass[11pt]{article}
\usepackage[left=1in,top=1.5in,right=1in,bottom=0.5in]{geometry}

\begin{document}

\begin{tabular}{|l|l|l|l|l|}
\hline
function   & domain           & description        & calculated from & example
\\
\hline
$\sigma_q$ & $(Q')$           & statistic          & $q$ / $\sigma_q$    & moment
\\
$\sigma_r$ & $(R')$           & statistic          & $r$ / $\sigma_r$    & moment
\\
\hline
$\phi$     & $(q \times r)$   & pairwise           & $(q, r)$            & kernel
\\
$\lambda$  & $(q \times R')$  & local accumulation & $\phi$              & local kernel sum
\\
$\rho$     & $(q \times R)$   & result per query   & $\lambda$ / $\delta$ & kernel sum
\\
\hline
$\delta$   & $(Q' \times R')$ & delta              & $(Q', R')$          & kernel sum refinement
\\
$\mu$      & $(Q' \times R)$  & mass result        & $\delta$ / $\rho$ / $\mu$ & kernel sum bounds, NN distance
\\
$\gamma$   & $(Q \times R)$   & global stat        & $\delta$ / $\phi$ (?$\mu$/$\lambda$?) & 2-point correlation
\\
\hline
$H$        & $(Q' \times R')$ & heuristic          & $\sigma_q$/$\sigma_r$/$\mu$ & bounding-box distance
\\
$S$        & $(Q' \times R')$ & prune              & $\sigma_q$/$\sigma_r$/$\mu$ & bound distance $\geq$ NN distance
\\
\hline
\end{tabular}

I think I've characterized data dependence.  Neglecting the practical
considerations (such as difference between points, info, and bounds), the above table
shows all the theoretical "functions" or rather "values" that we need to
calculate and update.

Notation: $Q$ is entire query tree, $Q'$ is a query node, $q$ is a query point.
I'm using these to represent points/nodes/trees, but in the "domain"
field I really technically mean "the space of" nodes/points/trees.

Most of it is straight-forward bottom-up propagation.
Of most interest is delta, however -- it is the only top-down computation, because it is
an estimation.

Basically, $\delta$ is an estimated change to bounding information.
In the case a prune occurs, that means that $\delta$ is in fact an exact computation, and
its values just need to lazily be propagated to children at some time.  Delta only would
need to be stored per-node (if that), and NOT per pair, even in breadth-first expansion.
Now this is a slight kicker -- if delta is zero, that implies $S$ = 1, but additionally,
for approximation purposes, $S$ may be *close* to zero.

In the case a prune doesn't occur, and $\delta$ is non-trivial, then $\delta$ would have
to be stored per $Q', R'$ pair for breadth-first or global-best-first.

\end{document}

